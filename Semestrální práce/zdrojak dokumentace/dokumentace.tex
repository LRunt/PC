\documentclass[12pt, a4paper]{article}

\usepackage [utf8]{inputenc}
\usepackage [IL2]{fontenc}
\usepackage [czech]{babel}
\usepackage{graphicx}
\usepackage[numbib]{tocbibind}
\usepackage{hyperref}
\usepackage{wrapfig}
\usepackage{pdfpages}
\usepackage{url}
\graphicspath{{d:/zzz/}}
\newcommand{\Break}{\State \textbf{break} }

\title{\includegraphics[width=10cm]{FAV_cmyk}

{\huge Semestrální práce z KIV/PC}

\vspace{0.5cm}
{\LARGE PŘEBARVOVÁNÍ SOUVISLÝCH OBLASTÍ VE SNÍMKU}
\vspace{1cm} 

\Large Lukáš Runt (A20B0226P)
\vspace{0.5cm} 

\large \itshape lrunt@students.zcu.cz
}
\date{\vspace{6cm} \today}

\begin{document}

\begin{titlepage}
\clearpage\maketitle
\thispagestyle{empty}
\end{titlepage}
\tableofcontents \newpage

\section{Zadání}
Naprogramujte v ANSI C přenositelnou {\bf konzolovou aplikaci}, která provede v binárním digitálním obrázku (tj. obsahuje jen černé a bílé body) {\bf obarvení souvislých oblastí} pomocí níže uvedeného algoritmu {\it Connected-Component Labeling} z oboru počítačového vidění. Vaším úkolem je tedy implementace tohoto algoritmu a funkcí rozhraní (tj. načítání a ukládání obrázku, apod.). Program se bude spouštět příkazem

\vspace{0.5cm}
\centerline{ \texttt{ccl.exe <input-file[.pgm]> <output-file>}}
\vspace{0.5cm}

\noindent Symbol \texttt{<input-file>} zastupuje jméno vstupního souboru s binárním obrázkem ve formátu {\it Portable Gray Map}, přípona souboru nemusí být uvedena; pokud uvedena není, předpokládejte, že má soubor příponu \texttt{.pgm} Symbol \texttt{<output-file>} zastupuje jméno výstupního souboru s obarveným obrázkem, který vytvoří vaše aplikace. Program tedy může být během testování spuštěn například takto:

\vspace{0.5cm}
{ \texttt{...\textbackslash>ccl.exe e:\textbackslash images\textbackslash img-inp-01.pgm e:\textbackslash images\textbackslash img-res-01.pgm}}
\vspace{0.5cm}

Úkolem programu je tedy vytvořit výsledný souboru s obarveným obrázkem uvedeném umístění a s uvedeným jménem. Vstupní i výstupní obrázek bude uložen ve formátu PGM. 

Testujte, zda je vstupní obraz skutečně černobílý. Musí obsahovat pouze pixely s hodnotou \texttt{0x00} a \texttt{0xFF}. Pokud tomu tak není, vypište krátké chybové hlášení (anglicky) a oznamte chybu operačnímu prostředí pomocí nenulového návratového kódu.

\vspace{0.5cm}
\noindent Hotovou práci odevzdejte v jediném archivu typu ZIP prostřednictvím automatického odevzdávacího a validačního systému. Postupujte podle instrukcí uvedených na webu předmětu. Archiv nechť obsahuje všechny zdrojové soubory potřebné k přeložení programu, {\bf makefile} pro Windows i Linux (pro překlad v Linuxu připravte soubor pojmenovaný \texttt{makefile} a pro Windows \texttt{makefile.win}) a dokumentaci ve formátu PDF vytvořenou v typografickém systému \TeX, resp. \LaTeX. Bude-li některá z částí chybět, kontrolní skript Vaši práci odmítne.

\vspace{0.5cm}
\noindent Celé zadání je dostupné na adrese: \url{http://www.kiv.zcu.cz/studies/predmety/pc/data/works/sw2021-02.pdf}


\section{Analýza úlohy}
Zadaný problém lze rozdělit na několik dílčích podproblémů, a to na kontrolu vstupních argumentů, načítání a vytváření pgm souboru a následně algoritmus, který přebarví souvislé oblasti.
\subsection{Formát PGM souboru}
\label{s:2}
Na vstupu a výstupu programu je soubor ve formátu PGM. Formát tohoto souboru je následující:

\begin{wrapfigure}{l}{0.4\textwidth}
    \includegraphics{formát pgm}
    \caption{PGM soubor}
\end{wrapfigure}
Každý PGM soubor obsahuje "Magické číslo" pro identifikaci typu souboru (v našem případě 'P5', tedy binární šedotónová data), bílý znak (může být mezera, TAB, CR, LF ...), šířku (počet sloupců, řetězcem znaků), bílý znak, výšku (počet řádků, řetězcem znaků), bílý znak, index nejvyšší hodnoty šedé (řetězcem znaků), bílý znak, byty (v jazyce C datový typ \texttt{unsigned char}) představující hodnoty jednotlivých pixelů.

\subsection{Algoritmus Přebarvování souvislých oblastí} \label{algoritmus}
Přebarvování souvislých oblastí (Connected-Component Labeling, CCL) je dvouprůchodový algoritmus z oblasti počítačového vidění. Jeho vstupem je binární obrázek (takový který obsahuje jen černé a bílé pixely). Bílé pixely představují objekty, které ten se tento algoritmus pokouší izolovat a označit různými barvami, resp. hodnotami intenzity šedé barvy (tedy tzv. labely, česky štítky, značkami). Černé pixely pak představují pozadí.

\subsubsection{První průchod}
Procházíme obrázek po řádcích. Každému nenulovému pixelu ne souřadnicích $[i, j]$ přiřadíme hodnotu podle hodnoty jeho sousedních pixelů, pokud nenulové sousední pixely existují (poloha sousedů je daná maskou na obrázku č. 1). Všechny sousední pixely dané maskou již byly obarveny v předchozích krocích (to je zajištěno tvarem masky).
\begin{itemize}
  \item Jsou-li všechny sousední pixely součástí pozadí (mají hodnotu 0x00), přiřadíme pixelu $[i, j]$ dosud nepřidělenou hodnotu, nebo-li novou barvu.
  \item Má-li právě jeden ze sousedních pixelů nenulovou hodnotu, přiřadíme obarvovanému pixelu hodnotu nenulového sousedního pixelu.
  \item Je-li více sousedních pixelů nenulových, přiřadíme obarvovanému pixelu hodnotu kteréhokoli nenulového pixelu ze zkoumaného okolí. Pokud byly hodnoty pixelů v masce různé (došlo k tzv. {\it kolizi barev}), zaznamenáme ekvivalenci dvojic do zvláštní datové struktury - tabulky ekvivalence barev.
\end{itemize}
\begin{figure}[h!]
\centering 
\includegraphics{maska} 
\caption{Maska pro přebarvování \cite{Zadani}}
\label{maska}
\end{figure}

\subsubsection{Druhý průchod}
Po prvním průchodu celého obrázku jsou všechny pixely náležející oblastem (objektům) obarveny, některé oblasti jsou však obarveny více barvami (díky kolizím). Proto musíme znova projít celý obrázek a podle informací o ekvivalenci barev (z tabulky ekvivalence barev získané v průběhu 1. průchodu) přebarvíme pixely těchto oblastí. Z množiny kolizních barev je možné nějak vybrat jednu (první, poslední, náhodnou) nebo opět postupovat při přiřazování barev "od začátku" a jako novou barvu použít index množiny (nesmí být samozřejmě nulový, protože barva \texttt{0x00} představuje pozadí).

Po tomto kroku odpovídá každé oblasti označení (obarvení) jedinou, v jiné oblasti se nevyskytující hodnotou (barvou).

\subsection{Zaznamenání kolizí}
Při prvním průchodu je potřeba si zaznamenávat kolize, pokud je přítomno v masce více rozdílných barev. Tyto kolize lze zaznamenávat pomocí několika níže uvedených metod.

\subsubsection{Množina}
Množina je abstraktní datový typ \cite{Set}, který je schopen uložit určité hodnoty bez jakéhokoliv pořadí a bez opakujících se hodnot. Na rozdíl od pole jsou množiny neuspořádané a neindexované. V informatice je teorie množin užitečná, pokud potřebujeme shromáždit data a nezáleží nám na jejich násobnosti nebo pořadí, což je právě náš případ. V případě kolize by se prvky přidávali do množin a při druhém průchodu by se kontrolovalo ve které množině se prvek nachází, následně by se bral nejmenší prvek.

\subsubsection{Spojový seznam}
Spojový seznam\cite{LinkedList} je dynamická datová struktura, určená k ukládání dat předem neznámé délky. Základní stavební jednotkou spojového seznamu je uzel, který vždy obsahuje ukládanou hodnotu a ukazatel na následující prvek. V našem případě by toto šlo implementovat tak, že by měl jeden uzel více ukazatelů na více uzlů, se kterými je v kolizi, nebo by se mohl při kolizi vložit prvek do náležitého seznamu, pokud by ještě prvek v seznamu nebyl.

\begin{figure}[h!]
\centering 
\includegraphics[width=10cm]{linkedList} 
\caption{Grafické znázornění spojového seznamu \cite{LinkedList}}
\end{figure}

\subsubsection{Pole}\label{pole}
Tento způsob je modifikovaný způsob se spojovými seznamy. Jedná se o pole, kde každý index, kromě indexu 0, znamená index barvy. Obsah indexu pak znamená v případě, že není zrovna roven -1 (-1 znamená, že index nikam neukazuje), ukazatel na index další barvy, se kterou je v kolizi. Kolize se pak vyhodnocují procházením po indexech, dokud není na indexu pole -1, to znamená, že jsme na nejmenší hodnotě barvy, se kterou je barva, kterou vyhodnocujeme, v kolizi. 

\begin{figure}[h!]
\centering 
\includegraphics[width=10cm]{ukazka pole} 
\caption{Grafické znázornění logiky pole}
\end{figure}

\section{Popis implementace}
Program obsahuje zdrojové soubory \texttt{main.c}, který zajišťuje hlavní běh programu, \texttt{pgm.c} zajišťuje práci s pgm souborem a \texttt{matrix.c} zajišťuje uložení dat pgm souboru.

\subsection{Kontrola a úprava argumentů}
Při spuštění programu se musejí zadat dva argumenty, program tedy kontroluje zda byl zadán správný počet argumentů. Jelikož uživatel nemusí zadávat typ vstupního souboru (přípona \texttt{.pgm}) musí být zajištěno, aby se soubor při čtení našel. Program tedy k argumentu tuto příponu přidá, pokud chybí, pomocí funkce \texttt{strncat}.

\subsection{Struktura pgm}
Struktura pgm reprezentuje soubor pgm. Tato struktura uchovává informace o typu souboru (náš program přijímá pouze P5), šířku, výšku a kontrast obrázku a nakonec data, která jsou uložena ve struktuře matice \ref{matice}.

\subsubsection{Načítání ze souboru}
Pro načítání dat ze souboru slouží funkce \texttt{read\_file}, která kontroluje, zda přišel soubor ve správném formátu, dále ukládá informace o souboru (šířka, výška, kontrast) a poté se vytvoří matice, která  se naplní hodnotami. Hodnoty se do matice plní znak po znaku, zároveň se kontroluje, zda jsou data jen černobílá (\texttt{0x00} nebo \texttt{0xFF}). Pokud data nejsou černobílá, program skončí výjimkou.

\subsubsection{Algoritmus obarvení dat}
Algoritmus má za úkol funkce \texttt{data\_coloring}. Jedná se o dvojprůchodový algoritmus \cite{CCL} \ref{algoritmus}. Pro rychlejší běh programu jsou pro první průchod vytvořeny tři cykly. První pro levý horní roh (tj. kontroluje se jen hodnota levého rohu), druhý pro první řádek kromě levého rohu (tj. kontroluje se jen prvek "nalevo") a nakonec pro zbylá data, kde se musí kontrolovat všechna data podle masky. U posledního cyklu se musí také kontrolovat kolize, které se zaznamenávají do pole \ref{pole}, neboť je to asi paměťově nevýhodnější. Pole má na začátku danou počáteční velikost a postupně se zvětšuje podle potřeby (pomocí znovurozdělení paměti). Po dokončení prvního průchodu se provede průchod druhý, tedy přebarvení podle zaznamenaných kolizí. 

\subsubsection{Zaznamenávání kolizí}
O zaznamenávání kolizí se stará funkce \texttt{detect\_colision}, která prochází vždy jednu část masky obr.\ref{maska} a kontroluje jí s aktuální minimální hodntou, která byla zjištěna v jiných částech masky. Do pole na zaznamenání kolizí se pak vždy ukládá nejmenší hodnota. Musí se však také kontrolovat, zda už v poli není uložena menší hodnota. Můžou nastat celkem 3 stavy: nejmenší prvek, je nová hodnota (právě zjištěná hodnota), stará hodnota (hodnota z jiné části masky), nebo hodnota která je uložená v poli o zaznamenávání kolizí na pozici nové nebo staré hodnoty. Ukládání o kolizích se pak provádí tak, že se prochází celé pole o kolizích (\texttt{color\_equivalence}) a hledají se hodnoty u kterých byla zjištěna kolize, jejich hodnota se pak mění na nejmenší zjištěnou hodnotu.

\subsubsection{Vytváření souboru}
Pro vytváření souboru slouží funkce \texttt{make\_file}. Tato funkce vytváří pgm soubor dle stanoveného formátu \ref{s:2}. Data se násobí hodnotou \texttt{0x30} a následně se do souboru uloží zbytek z dělení hodnotou \texttt{0xFF}, aby byly oblasti lépe rozeznatelné.
\begin{figure}[h!]
\centering 
\includegraphics[width=5cm]{zvýraznění} 
\caption{Výsledný obrázek bez provedení zvýraznění}
\end{figure}

\subsubsection{Uvolňování paměti}
Paměť, kterou zabírá pgm struktura, uvolňuje funkce \texttt{free\_pgm}. 

\subsection{Struktura matrix} \label{matice}
Struktura matrix představuje matici, která je implementována jednorozměrným polem. Struktura dále uchovává údaj o počtu sloupců (šířce) a řádků (výšce). Dále jsou implementovány tři funkce, a to: \texttt{create\_matrix} vytvářející matici, \texttt{print\_matrix} vypisující matici a \texttt{free\_matrix} uvolňující paměť.

\section{Uživatelská příručka}

\subsection{Sestavení}
Pro snadné sestavení je připraven \texttt{makefile}, který funguje na operačním systému Linux a v případě správné konfigurace i na Windows. Sestavení se provede po zavolání příkazu \texttt{make} v kořenovém adresáři. Předpokladem je nainstalovaný {\it GCC} a nástroj {\it make}. 

Pro operační systém Windows je připraven ještě poněkud typičtější \texttt{makefile.win}. Překlad lze provést příkazem \texttt{make -f makefile.win}. Předpokladem je nainstalovaný překladač {\it Microsoft C/C++} a nástroj {\it make} Příkazy pro přeložení se mohou lišit podle použitého překladače a nastavení.

\subsection{Spuštění}
Program lze spustit pomocí příkazu:

\vspace{0.5cm}
\centerline{ \texttt{ccl.exe <input-file[.pgm]> <output-file>}}
\vspace{0.5cm}

\noindent Kde \texttt{input-file} je cesta k pgm souboru, který obsahuje jen hodnoty \texttt{0x00} a \texttt{0xFF}. Při spuštění se může soubor zadat s příponou \texttt{.pgm} i bez. \texttt{output-file} pak obsahuje cestu a jméno výstupního souboru, který se může zadat s příponou \texttt{.pgm} i bez. 
\begin{figure}[h]
\centering 
\includegraphics{prikladspusteni}
\caption{Příklad spuštění}
\end{figure}

Formát vstupního i výstupního souboru je typu pgm (formát je detailně popsán viz.\ref{s:2}). Přebarvení obrázku by mělo vypadat následovně:
\begin{figure}[h]
  \centering
  \begin{minipage}[b]{0.45\textwidth}
    \includegraphics[width=\textwidth]{w2test.pdf}
    \caption{Příklad vstupu}
  \end{minipage}
  \hfill
  \begin{minipage}[b]{0.45\textwidth}
    \includegraphics[width=\textwidth]{test.pdf}
    \caption{Příklad výstupu}
  \end{minipage}
\end{figure}

\newpage
\section{Závěr}
Byl vytvořen program na přebarvování souvislých oblastí. Myslím, že z hlediska paměťové náročnosti se jedná o celkem slušné řešení, které se moc vylepšit nedá, avšak to samé nelze říci o rychlosti. Myslím, že je v mém programu přehnaný počet podmínek a cyklů, bez kterých by však program nefungoval správně. K zlepšení rychlosti by bylo potřeba se na problém podívat z jiného úhlu, a to hlavně na implementování jiného způsobu ukládání kolizí, při této implementaci by nejspíše utrpěla paměťová složitost. Další škraloup na mé práci je horší modifikovatelnost. Metody by možná mohli být kratší a obecnější. Například algoritmus by se dal rozdělit na první a druhý průchod, nebo vylepšit načítání souboru, aby šel lehce implementovat kód pro jiný typ pgm souboru (např. P2).

Celkovou práci hodnotím pozitivně, neboť jsem si vyzkoušel práci s jazykem C. Byl to pro mne nepopsatelný zážitek, který mě studijně obohatil a posunul o krok blíže k praktickým aplikacím teoreticky získaných vědomostí. Získal jsem prstoklad na klávesnici, jak kdybych hrál na klavír, celou noc jsem programoval, byl jsem jako netopýr. Myslím, že mi "Céčko" změnilo pohled na svět a můj život už nebude takový jako předtím. Přinejmenším pokaždé, když půjdu s dívkou na rande, dívat se na noční oblohu, uvidím místo hvězd miliony pointerů. Od té doby co umím alespoň trošku programovat v tomto skvělém jazyce, si připadám, jako kdybych měl oproti normálním programátorům tajnou super schopnost. Takže, když se mne někdo ode dneška zeptá na znalost jazyků, řeknu, že umím v Céčku, javě, sql a basicu.

\begin{thebibliography}{9}

\bibitem{Zadani}
	Podrobné zadání úlohy č.2,
	\url {http://www.kiv.zcu.cz/studies/predmety/pc/data/works/sw2021-02.pdf},
	Kamil Ekštein,
	2021

\bibitem{CCL}
	Connected-component labeling, 
	\url {https://en.wikipedia.org/wiki/Connected-component_labeling},
 	Wikipedia, 
	Wikimedia Foundation, 
	2021

\bibitem{Set}
	Set (abstract data type),
	\url {https://en.wikipedia.org/wiki/Set_(abstract_data_type)},
 	Wikipedia, 
	Wikimedia Foundation,
 	2021

\bibitem{LinkedList}
	Linked list, 
	\url {https://en.wikipedia.org/wiki/Linked_list},
 	Wikipedia,
	Wikimedia Foundation, 
	2022

\end{thebibliography}

\end{document}